\chapter{Конструкторский раздел}


\section{Аппроксимация трёхмерных объектов}

В программе будет генерироваться икосфера --- это многогранная сфера, состоящая из треугольников.

Алгоритм триангуляции сферы следующий.

\begin{enumerate}
	\item Задать число сечений (параллелей) и количество разбиений для каждого сечения (меридиан).
        \item На каждом сечении подсчитать углы поворота радиуса вектора для очередной точки грани по формулам следующей системы уравнений~(\ref{eq:angles}):
        \begin{equation}\label{eq:angles}
            \begin{cases}
	        \alpha = \frac{\pi}{k_{1}}, \\
                \beta = \frac{2 \cdot \pi}{k_{2}},
            \end{cases}
        \end{equation}
        \noindent где $k_{1}$, $k_{2}$ --- число сечений и разбиений для каждого сечения.
	\item На каждом сечении найти нужные точки, разделив окружность на равные части.
	\item Соединив точки, получим сферу, разбитую на треугольники.
\end{enumerate}

Координаты вершин аппроксимирующих треугольников можно рассчитать по формулам следующей системы уравнений~(\ref{eq:triangle}):
    \begin{equation}\label{eq:triangle}
        \begin{cases}
	    x = r * \sin(\alpha \cdot i) \cdot \cos(\beta \cdot j), i = \overline{0, k_{1}}, j = \overline{0, k_{2} - 1}\\
            y = r * \cos(\alpha \cdot i),\\
            z = r * \sin(\alpha \cdot i) \cdot \sin(\beta \cdot j),
        \end{cases}
    \end{equation}

\noindent где $r$ --- радиус сферы.


%\section{Структура преобразований трёхмерных координат}
%Экран обладает только двумя координатами, поэтому нужно выбрать способ для переноса трёхмерных объектов на двухмерное пространство экрана. Алгоритм приведения координат к нужному виду представлен на рисунке 3.1 в виде схемы последовательных преобразований.%

\section{Объединенный алгоритм удаления невидимых граней с использованием z-буфера и закраски по Гуро}

На рисунке \ref{img:zbuf} приведена схема алгоритма z-буфера. Совместно с алгоритмом z-буфера применяется закраска по Гуро.

\img{180mm}{zbuf}{Схема алгоритма z-буфера с закраской по Гуро}


\section{Алгоритм перемещения сферы}

Для нахождения точек орбиты сферы используется метод численного интегрирования Верле~\cite{verle}. Это алгоритм позволяет определить следующее положение тела, зная текущее и предыдущее положения без использования скорости.

Раскладывается вектор местоположения точки $\overrightarrow{r}(t)$ в моменты времени $(t + \bigtriangleup{t})$ и $(t - \bigtriangleup{t})$ по формуле Тейлора~(\ref{eq:teilor}):
    \begin{equation}\label{eq:teilor}
        \begin{cases}
	    \overrightarrow{r}(t + \bigtriangleup{t}) = \overrightarrow{r}(t) + \overrightarrow{v}(t) \cdot \bigtriangleup{t} + \frac{\overrightarrow{a}(t)}{2} \cdot \bigtriangleup{t}^2 + \frac{\overrightarrow{b}(t)}{6} \cdot \bigtriangleup{t}^3+ O(\bigtriangleup{t}^4),\\
            \overrightarrow{r}(t - \bigtriangleup{t}) = \overrightarrow{r}(t) - \overrightarrow{v}(t) \cdot \bigtriangleup{t} + \frac{\overrightarrow{a}(t)}{2} \cdot \bigtriangleup{t}^2 - \frac{\overrightarrow{b}(t)}{6} \cdot \bigtriangleup{t}^3+ O(\bigtriangleup{t}^4),
        \end{cases}
    \end{equation}

\noindent где $\overrightarrow{v}(t)$ --- вектор скорости, $\overrightarrow{a}(t)$ --- вектор ускорения, $\overrightarrow{b}(t)$ --- производная ускорения по времени.

Затем складываются полученные формулы и выражается $\overrightarrow{r}(t + \bigtriangleup{t})$ по формуле~(\ref{eq:r}):
    \begin{equation}\label{eq:r}
	    \overrightarrow{r}(t + \bigtriangleup{t}) = 2 \cdot \overrightarrow{r}(t) - \overrightarrow{r}(t - \bigtriangleup{t}) + \overrightarrow{a}(t) \cdot \bigtriangleup{t}^2 + O(\bigtriangleup{t}^4).
    \end{equation}

Небесные сферы двигаются по действием силы гравитационного взаимодействия, которая расчитывается по формуле~(\ref{eq:f}):
    \begin{equation}\label{eq:f}
	    \overrightarrow{F} =  G \cdot \frac{m_{1} \cdot m_{2}}{r^3} \cdot \overrightarrow{r}(t),
    \end{equation}

\noindent где $\overrightarrow{r}(t)$ --- вектор, соединяющий звезды, $m_{1}$, $m_{2}$ --- массы звезд, $G$ --- гравитационная постоянная.

Тогда из второго закона Ньютона выражается ускорение по формуле~(\ref{eq:a}):
    \begin{equation}\label{eq:a}
	    \overrightarrow{a}(t) = \frac{\overrightarrow{F}(t)}{m}.
    \end{equation}

Возникает проблема с поиском положения в самый первый момент и последующий за ним. Предыдущему положению можно задать начальное условие (положение старта), а текущее вычисляется по формуле равноускоренного движения~(\ref{eq:dvig}):
    \begin{equation}\label{eq:dvig}
	    \overrightarrow{r}(t + \bigtriangleup{t}) = \overrightarrow{r}(t) + \overrightarrow{v}(t) \cdot \bigtriangleup{t} + \frac{\overrightarrow{a}(t)}{2} \cdot \bigtriangleup{t}^2.
    \end{equation}


\section{Вывод}

В данном разделе были описаны алгоритм передвижения сфер, способ моделирования многогранной сферы, приведена схема алгоритма, объединяющего z-буфер и закраску по Гуро.

