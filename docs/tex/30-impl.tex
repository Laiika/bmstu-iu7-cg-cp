\chapter{Технологический раздел}

\section{Выбор языка и среды программирования}

В качестве языка программирования был выбран С++, по следующим причинам:

\begin{itemize}[label=---]
	\item поддержка ООП;
        \item благодаря своей вычислительной мощности язык обеспечивает высокую скорость исполнения кода;
	\item в стандартной библиотеке имеется множество готовых контейнеров и алгоритмов.
\end{itemize}

В качестве среды программирования был выбран <<QtCreator>>, так как:

\begin{itemize}[label=---]
	\item в рамках фреймворка Qt представлена многофункциональная и производительная графическая библиотека;
	\item данная среда бесплатна для студентов.
\end{itemize}


\section{Структура программы}

Структура программы представлена на рисунке 3.1 в виде схемы последовательных преобразований.

\img{15mm}{struct}{Структура программы}

Разработанная программа состоит из следующих классов.
  
\begin{enumerate}
	\item Классы объектов сцены:
	\begin{itemize}
		\item Camera --- класс камеры с возможностью перемещения по сцене;
		\item Sphere --- класс трёхмерных объектов (сфер), содержащий характеристики сфер и их орбиты.
	\end{itemize}
        \item Класс для работы с моделями сфер:
	\begin{itemize}
		\item Mesh --- класс полигональной сетки, описывающий представление трёхмерного объекта в программе и методы работы с ним.
	\end{itemize}
	\item Вспомогательные классы сцены:
	\begin{itemize}
		\item Scene --- класс, содержащий набор объектов;
            \item Painter --- класс отрисовки сцены.
	\end{itemize}
	\item Класс интерфейса пользователя:
	\begin{itemize}
		\item MainWindow --- класс главного окна сцены. 
	\end{itemize}	
\end{enumerate}

На рисунке \ref{img:shema} приведена диаграмма классов.

\img{80mm}{shema}{Диаграмма классов}


\section{Реализации алгоритмов}

В листинге \ref{lst:orbit} представлен алгоритм расчета точек орбиты сферы, в листинге \ref{lst:zbuff} --- алгоритм z-буфера, совмещенный с закраской по Гуро.


\begin{center}
\captionsetup{justification=raggedright,singlelinecheck=off}
\begin{lstlisting}[label=lst:orbit,caption=Алгоритм расчета точек орбиты сферы]
void Sphere::calc_trajectory(Point3D& s2, Point3D& vs1, Point3D& vs2, double m2)
{
    Point3D fs, es, s1 = center;
    double m1 = m;
    double rasts = sqrt((s1.x - s2.x) * (s1.x - s2.x) + (s1.z - s2.z) * (s1.z - s2.z));
    double dt = 0.025;
    points.clear();

    double xp2 = s2.x, xp1 = s1.x;
    double zp2 = s2.z, zp1 = s1.z;
    double xc1 = xp1 + vs1.x * dt - (dt * dt * 0.5 * G * m2 * (s1.x - s2.x)) / (rasts * rasts * rasts);
    double xc2 = xp2 + vs2.x * dt + (dt * dt * 0.5 * G * m1 * (s1.x - s2.x)) / (rasts * rasts * rasts);
    double zc1 = zp1 + vs1.z * dt - (dt * dt * 0.5 * G * m2 * (s1.z - s2.z)) / (rasts * rasts * rasts);
    double zc2 = zp2 + vs2.z * dt + (dt * dt * 0.5 * G * m1 * (s1.z - s2.z)) / (rasts * rasts * rasts);

    s2.x = xc2;
    s2.z = zc2;
    s1.x = xc1;
    s1.z = zc1;

    for (int i = 0; i < 207; i++)
    {
        points.push_back(s1);

        rasts =  sqrt((s1.x - s2.x) * (s1.x - s2.x) + (s1.z - s2.z) * (s1.z- s2.z));
        es.x = (s1.x - s2.x) / rasts;
        es.z = (s1.z - s2.z) / rasts;
        fs.x = (G * m1 * m2 * es.x) / (rasts * rasts);
        fs.z = (G * m1 * m2 * es.z) / (rasts * rasts);

        s1.x = 2 * xc1 - xp1 - (fs.x) * dt * dt / m1;
        s1.z = 2 * zc1 - zp1 - (fs.z) * dt * dt / m1;
        xp1 = xc1;
        xc1 = s1.x;
        zp1 = zc1;
        zc1 = s1.z;

        s2.x = 2 * xc2 - xp2 + (fs.x) * dt * dt / m2;
        s2.z = 2 * zc2 - zp2 + (fs.z) * dt * dt / m2;
        xp2 = xc2;
        xc2 = s2.x;
        zp2 = zc2;
        zc2 = s2.z;
    }
    cur = 0;
}
\end{lstlisting}
\end{center}

\begin{center}
\captionsetup{justification=raggedright,singlelinecheck=off}
\begin{lstlisting}[label=lst:zbuff,caption=Алгоритм z-буфера с закраской по Гуро]
void Painter::textureSpTriangle(Mesh mesh, int a, int b, int c, std::vector<QPoint> points, std::vector<QPoint> texturePoints, QImage *img, QImage *img2, double light, Point3D p)
{
    double u,v,w;
    QColor kolor;
    double Xa = points[0].x(),Xb = points[1].x(),Xc = points[2].x();
    double Ya = points[0].y(),Yb = points[1].y(),Yc = points[2].y();

    double div = (Xb - Xa)  * (Yc - Ya) - (Yb - Ya) * (Xc - Xa);
    if(div == 0.0)
        return;

    Point3D Na = Point3D(mesh.Tnodes[a].x - p.x, mesh.Tnodes[a].y - p.y, mesh.Tnodes[a].z - p.z);
    Point3D Nb = Point3D(mesh.Tnodes[b].x - p.x, mesh.Tnodes[b].y - p.y, mesh.Tnodes[b].z - p.z);
    Point3D Nc = Point3D(mesh.Tnodes[c].x - p.x, mesh.Tnodes[c].y - p.y, mesh.Tnodes[c].z - p.z);
    Point3D pa = mesh.Tnodes[a],pb = mesh.Tnodes[b],pc = mesh.Tnodes[c];
    double Ia = countStarLightIntense(Na, pa);
    double Ib = countStarLightIntense(Nb, pb);
    double Ic = countStarLightIntense(Nc, pc);

    Point3D wall = mesh.countNormalVector(mesh.Tnodes[a],mesh.Tnodes[b], mesh.Tnodes[c]);

    QPoint A = mesh.points[a], B = mesh.points[b],C = mesh.points[c];
    if (A.y() > B.y())
    {
        std::swap(A, B);
        std::swap(Ia, Ib);
    }
    if (A.y() > C.y())
    {
        std::swap(A, C);
        std::swap(Ia, Ic);
    }
    if (B.y() > C.y())
    {
        std::swap(B, C);
        std::swap(Ib, Ic);
    }

    int xmin = (int)minim(Xa, Xb, Xc), ymin = (int)minim(Ya, Yb, Yc);
    int xmax = (int)maxim(Xa, Xb, Xc), ymax = (int)maxim(Ya, Yb, Yc);
    for(int y = ymin; y < ymax; y++)
    {
        bool half2 = (y - ymin) > B.y() - A.y() || B.y() == A.y();
        int h = half2 ? C.y() - B.y() : B.y() - A.y();
        float alpha = (float)(y - ymin) / (ymax - ymin);
        float betta = (float)(y - ymin - (half2 ? B.y() - A.y() : 0)) / h;

        QPoint minp = A + (C - A) * alpha;
        QPoint maxp = half2 ? B + (C - B) * betta : A + (B - A) * betta;
        double Id, If;
        Id = Ia + (Ic - Ia) * alpha;
        If = half2 ? Ib + (Ic - Ib) * betta : Ia + (Ib - Ia) * betta;

        if (minp.x() > maxp.x())
        {
            std::swap(minp, maxp);
            std::swap(Id, If);
        }
        minp.setX(std::max(xmin, minp.x()));
        maxp.setX(std::min(xmax, maxp.x()));
        int d = maxp.x() - minp.x(), f = minp.x(), l = maxp.x();
        for(int x = f; x < l; x++)
        {
            double Z = (-wall.x * (x - mesh.Tnodes[a].x) + -wall.y * (y - mesh.Tnodes[a].y))/wall.z + mesh.Tnodes[a].z;
            if (zBuffer[x][y] < Z)
            {
                v = ((x - Xa) * (Yc - Ya) - (y - Ya) * (Xc - Xa))/div;
                w = ((Xb - Xa) * (y - Ya) - (Yb - Ya) * (x - Xa))/div;
                u = 1.0 - v - w;
                if(u < 0 || u > 1 || v < 0 || v > 1 || w < 0 || w > 1) continue;

                double xT = u * texturePoints[0].x() + v * texturePoints[1].x() + w * texturePoints[2].x();
                double yT = u * texturePoints[0].y() + v * texturePoints[1].y() + w * texturePoints[2].y();
                if(yT >= img->height()) yT = img->height() -1;
                if(yT < 0) yT = 0;
                if(xT >= img->width()) xT = img->width() -1;
                if(xT < 0) xT = 0;

                QRgb rgb = img->pixel(xT, yT);
                kolor = QColor(qRed(rgb), qGreen(rgb), qBlue(rgb), qAlpha(rgb));

                float phi = l == f ? 1. : (float)(x - f) / (float)(l - f);
                double I = Id + (If - Id) * phi;
                light = I;
                setColor(x, y, kolor, light, img2);
                zBuffer[x][y] = Z;
            }
        }
    }
}
\end{lstlisting}
\end{center}


\section{Описание интерфейса}
На рисунке \ref{img:screen} представлен интерфейс программы.

\img{130mm}{screen}{Интерфейс программы}

Функции представленных кнопок следующие:

\begin{itemize}[label=---]
	\item кнопки <<Запустить/Остановить>> позволяют запустить/остановить моделирование движения сфер;
	\item кнопки <<Вправо/Влево>> позволяют перемещать камеру по оси ОX, то есть вправо/влево;
        \item кнопки <<Вверх/Вниз>> позволяют перемещать камеру по оси ОY, то есть вверх/вниз.
\end{itemize}

Так же для перемещения камеры вправо/влево можно использовать кнопки <<E>> И <<Q>> соответственно, для перемещения камеры вверх/вниз --- кнопки <<A>> и <<D>>.


\section{Вывод}
В этом разделе был выбран язык программирования и среда разработки, рассмотрена диаграмма основных классов, приведены реализации алгоритмов и разобран интерфейс приложения.