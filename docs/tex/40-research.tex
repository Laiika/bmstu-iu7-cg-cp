\chapter{Экспериментальный раздел}

\section{Технические характеристики}

Технические характеристики устройства, на котором выполнялось тестирование, следующие:

\begin{itemize}[label=---]
	\item операционная система Windows 11  64–bit;
	\item оперативная память 16 ГБ;
	\item процессор 2.40 ГГц Intel Core i5–1135G7 \cite{intel}.
\end{itemize}

Тестирование проводилось на ноутбуке, включенном в сеть электропитания. Во время тестирования ноутбук был нагружен только встроенными приложениями окружения, а также системой тестирования.


\section{Цель эксперимента}

Для программы, изображающей динамичную сцену, важна скорость отрисовки сцены. На скорость отрисовки  может повлиять изменение количества полигонов, аппроксимирующих объекты сцены. 

В рамках данной курсовой работы проведено исследование зависимости времени отрисовки сцены от количества полигонов, аппроксимирующих объекты сцены. Критерием измерения будет среднее время отрисовки кадра.


\section{План эксперимента}

В ходе эксперимента меняется количество полигонов: 100, 300, 500, 800, 1000, 1200, 1400. Каждый замер производится 10 раз, за результат выбирается среднее арифметическое времени отрисовки кадра. Для замера времени используется библиотека chrono.

Так как оценивается не только скорость синтезирования изображения, но
и быстродействие симуляции движения сфер, в эксперименте сферы будут находится в движении. Камера для всех экспериментов находятся в одинаковой позиции и остается неподвижной на протяжении всех замеров.


\section{Результат эксперимента}

Результаты эксперимента представлены в виде таблицы~\ref{tab:time} и графика~\ref{img:graph}.

\begin{table}[h]
	\begin{center}
		\caption{\label{tab:time} Зависимость времени отрисовки сцены от количества полигонов, аппроксимирующих сферы}
		\begin{tabular}{|c|c|}
			\hline
			Количество полигонов сферы & Время, мкс\\
			\hline
			100 & 42052\\
			\hline
			300 & 48833\\
			\hline
			500 & 53089\\
			\hline
			800 & 60611\\
			\hline
                1000 & 67723\\
			\hline
                1200 & 70069\\
			\hline
                1400 & 75393\\
			\hline
			
		\end{tabular}
	\end{center}
\end{table}

\img{85mm}{graph}{Зависимость времени отрисовки сцены от количества полигонов, аппроксимирующих сферы}


\section{Вывод}

Увеличение количества полигонов значительно влияет на скорость отрисовки сцены. При использовании сфер, каждая из которых представлена 1200 полигонов, становятся заметными задержки между кадрами. При меньших количествах данный эффект менее заметен.




