\chapter*{Введение}
\addcontentsline{toc}{chapter}{Введение}

Компьютерная графика используется практически во всех научных и инженерных дисциплинах для наглядности восприятия и передачи информации. Трехмерные изображения используются в медицине, картографии, геофизике, ядерной физике и других областях. 

Компьютерные модели физических явлений позволяют получить более полную информацию об изменяющихся физических величинах, построить соответствующие графики, траектории, увидеть исследуемые процессы в динамике. Последнее особо важно для формирования наглядного образа изучаемого явления.

В наше время астрономия приобрела особо важное значение. Без нее оказались бы невозможными многие достижения науки и техники, в том числе успехи человечества в освоении космоса.
Для решения фундаментальных проблем астрономии и астрофизики --- происхождение и эволюция звезд, строение галактик и история звездообразования, необходимо изучать звезды.

Кратная звездная система --- гравитационно-связанная система из нескольких звёзд с замкнутыми орбитами. Кратность звёздной системы ограничена. Невозможно создать долгоживущую систему из трех, четырех и более равноправных звезд. Устойчивыми оказываются только иерархические системы. 

Тройные звездные системы --- наиболее распространённый тип кратных систем. В соответствии с иерархическим принципом тройные звездные системы обычно состоят из пары близко расположенных звезд, вращающихся вокруг друг друга, и третьей, более отдаленной, которая вращается вокруг центра масс первых двух.

Целью данной курсовой работы является создание ПО для моделирования движения звезд и планеты в тройной звездной системе.

Для достижения поставленной цели необходимо решить следующие задачи:

\begin{itemize}[label=---]
    \item выделить объекты сцены и выбрать модель их представления;
    \item проанализировать и выбрать алгоритмы построения реалистичного трехмерного изображения;
    \item разработать физическую модель поведения объектов;
    \item разработать программу на основе выбранных алгоритмов; 
    \item провести исследование на основе разработанной программы.
\end{itemize}
