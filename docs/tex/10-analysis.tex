\chapter{Аналитический раздел}

\section{Описание моделей объектов сцены}

В качестве объектов визуализируемой сцены можно выделить следующие сущности.

\begin{enumerate}
	\item Камера --- наблюдатель. Характеризуется своим пространственным положением и направлением просмотра.
	\item Планета --- сфера. Характеризуется положением центра, массой и радиусом.
	\item Звезды --- сферы, излучающие свет во все стороны (источники света). Характеризуются положением центра, массой, радиусом и интенсивностью света.
\end{enumerate}

Для описания трёхмерных объектов существуют три модели: каркасная, поверхностная и объёмная. Для реализации поставленной задачи более подходящей моделью будет поверхностная. По сравнению с каркасной моделью она даст более реалистичное изображение. С другой стороны, ей требуется меньше памяти, чем объемной. 

Поверхностная модель может задаваться параметрическим представлением или полигональной сеткой.

\begin{enumerate}
	\item При параметрическом представлении поверхность можно получить при вычислении параметрической функции, что удобно при просчете поверхностей вращения.
	\item В случае полигональной сетки форма объекта задается совокупностью вершин, ребер и граней.
\end{enumerate}

Выбрано полигональное представление объектов, так как оно сочетается с основными алгоритмами для построения сцен.

В случае полигональной сетки форма объекта задается некоторой совокупностью вершин, ребер и граней, что позволяет выделить несколько способов представления.

\begin{enumerate}
	\item Вершинное представление --- при таком представлении вершины хранят указатели на соседние вершины. В таком случае для отрисовки нужно будет обойти все данные по списку, что может занимать достаточно много времени при переборе.
	\item Список граней --- при таком представлении объект хранится, как множество граней и вершин. В таком случае достаточно удобно производить различные манипуляции над данными.
	\item Таблица углов --- при таком представлении вершины хранятся в предопределенной таблице, такой, что обход таблицы неявно задает полигоны. Такое представление более компактное и более производительное для нахождения полигонов, однако, операции по замене достаточно медлительны.
\end{enumerate}

Наиболее подходящим в условиях поставленной задачи будет представление в виде списка граней, так как оно позволяет эффективно манипулировать данными, а также проводить явный поиск вершин грани и самих граней, которые окружают вершину.


\section{Анализ алгоритмов удаления невидимых линий и поверхностей}

В данной задаче акцент делается на частоту вывода изображения на экран. Движение объектов должно выглядеть как анимация, а не просто набор кадров. Поэтому алгоритм, который будет использоваться, должен работать быстро.

\subsection{Алгоритм, использующий z-буфер}
Данный алгоритм работает в пространстве изображений.

Используются два буфера: буфер кадра для запоминания цвета каждого пикселя и буфер глубины для запоминания глубины каждого пикселя. Вначале в z-буфер заносятся минимально возможные значения z, а буфер кадра заполняется значениями пикселя, описывающими фон. В процессе работы глубина (значение координаты z) каждого нового пикселя, который надо занести в буфер кадра, сравнивается с глубиной того пикселя, который уже занесён в z-буфер. Если это сравнение показывает, что новый пиксель расположен ближе к наблюдателю, чем пиксель, уже находящийся в буфере кадра, то новый пиксель заносится в буфер кадра. Кроме того, производится корректировка z-буфера: в него заносится глубина нового пикселя. Если же глубина нового пикселя меньше глубины хранящегося в буфере, то никаких действий производить не надо~\cite{rojers}.


\subsubsection*{Преимущества:}
\begin{itemize}[label=---]
	\item простота реализации;
	\item линейная зависимость от числа визуализируемых объектов.
\end{itemize}

\subsubsection*{Недостатки:}
\begin{itemize}[label=---]
	\item трудоемкость устранения лестничного эффекта;
	\item большой объем требуемой памяти.
\end{itemize}


\subsection{Алгоритм обратной трассировки лучей}
Данный алгоритм работает в пространстве изображения.

Алгоритм предлагает рассмотреть следующую ситуацию: через каждый пиксел изображения проходит луч, выпущенный из камеры, и программа должна определить точку пересечения этого луча со сценой. Первичный луч --- луч, выпущенный из камеры. На рисунке \ref{img:tras} представлена ситуация, когда первичный луч пересекает объект в точке H1.

\clearpage
\img{80mm}{tras}{Схема обратной трассировки лучей}

Для источника света определяется, видна ли для него эта точка. Чтобы это сделать, испускается теневой луч из точки сцены к источнику. Если луч пересек какой-либо объект сцены, то значит, что точка находится в тени, и ее не надо освещать. В обратном случае требуется рассчитать степень освещенности точки. Затем алгоритм рассматривает отражающие свойства объекта: если они есть, то из точки H1 выпускается отраженный луч, и процедура повторяется рекурсивно. Тоже самое происходит при рассмотрении свойств преломления~\cite{raytrac}. 

\subsubsection*{Преимущества:}
\begin{itemize}[label=---]
	\item линейная зависимость от количества объектов на сцене;
	\item позволяет передавать множество разных оптических явлений (отражение, преломление).
\end{itemize}

\subsubsection*{Недостатки:}
\begin{itemize}[label=---]
	\item производительность (большое количество вычислений для каждого луча).
\end{itemize}


\subsection{Алгоритм Робертса}

Данный алгоритм работает в объектном пространстве.

Алгоритм работает только с выпуклыми телами. Если тело изначально не выпуклое, то нужно его разбить на выпуклые составляющие.

Основные этапы~\cite{roberts}. 

\begin{enumerate}
	\item Подготовка исходных данных.
	\item Удаление ребер, экранируемых самим телом.
	\item Удаление ребер, экранируемых другими телами.
	\item Удаление линий пересечения тел, экранируемых самими телами и другими телами, связанными отношением протыкания.
\end{enumerate}

\subsubsection*{Преимущества:}
\begin{itemize}[label=---]
	\item точность вычислений (за счет работы в объектном пространстве).
\end{itemize}

\subsubsection*{Недостатки:}
\begin{itemize}[label=---]
	\item вычислительная трудоемкость алгоритма растет как квадрат количества объектов;
	\item тела должны быть выпуклыми (иначе нужно разбивать на выпуклые составляющие).
\end{itemize}


\subsection{Алгоритм Варнока}

Данный алгоритм работает в пространстве изображений. 

Алгоритм анализирует область на экране дисплея (окно) на наличие видимых элементов. Если в окне нет изображения, то оно просто закрашивается фоном. Если же в окне имеется элемент, то проверяется, достаточно ли он прост для визуализации. Если объект сложный, то окно разбивается на более мелкие, для каждого из которых выполняется тест на отсутствие или простоту изображения. Рекурсивный процесс разбиения может продолжаться до тех пор, пока не будет достигнут предел разрешения экрана~\cite{varnok}.

 
\subsubsection*{Преимущества:}
\begin{itemize}[label=---]
	\item эффективен для простых сцен (будет немного разбиений).
\end{itemize}

\subsubsection*{Недостатки:}
 \begin{itemize}[label=---]
	\item неэффективен для большого количества небольших по размеру многоугольников;
	\item неэффективен для большого числа пересечений объектов.
\end{itemize}
 
 
\subsection{Вывод}

Проанализировав данные алгоритмы, можно прийти к выводу, что наиболее подходящим алгоритмом для данной программы будет алгоритм z-буфера. Он позволит выполнять синтез изображения достаточно быстро.


\section{Анализ алгоритмов закрашивания}

\subsection{Простая закраска}

Вся грань закрашивается с одинаковой интенсивностью, которая вычисляется по формуле~(\ref{eq:simple}):

\begin{equation}\label{eq:simple}
	I = I_{0} \cdot k \cdot \cos(\alpha),
\end{equation}

\noindent где $I$ --- интенсивность света в точке, $I_{0}$ --- интенсивность источника света, $k$ --- коэффициент диффузного отражения, $\alpha$ --- угол падения луча.

Метод позволяет получать изображения, сравнимые по качеству с реальными объектами, лишь при выполнении следующих условий:

 \begin{itemize}[label=---]
	\item предполагается, что источник света находится в бесконечности;
	\item наблюдатель находится в бесконечности;
	\item закрашиваемая грань является реально существующей, а не полученной в результате аппроксимации поверхности.
\end{itemize}

\subsubsection*{Преимущества:}
\begin{itemize}[label=---]
	\item простой в реализации;
	\item небольшие требования к ресурсам.
\end{itemize}

\subsubsection*{Недостатки:}
\begin{itemize}[label=---]
	\item плохо подходит для тел вращения;
	\item плохо учитывает отраженный свет.
\end{itemize}


\subsection{Закраска по Гуро}

В данном методе используется интерполяция интенсивности~\cite{cul}. Рассматривается отдельная грань, вычисляются нормали к вершинам грани. Используя нормали в вершинах, вычисляется интенсивность каждой вершины, выполняется первая линейная интерполяция вдоль ребер. Вторая интерполяция (тоже линейная) выполняется, когда вычисляется интенсивности пикселей, расположенных на сканирующей строке. 

\subsubsection*{Преимущества:}
\begin{itemize}[label=---]
	\item подходит для фигур вращения, аппроксимированных полигонами;
	\item хорошо сочетается с диффузным отражением.
\end{itemize}

\subsubsection*{Недостатки:}
\begin{itemize}[label=---]
	\item могут быть потеряны ребра (получится плоское изображение).
\end{itemize}


\subsection{Закраска по Фонгу}

Данный метод схож с закраской по Гуро. Вместо интенсивности происходит интерполяция по значению самой нормали.

\subsubsection*{Преимущества:}
\begin{itemize}[label=---]
	\item улучшенная аппроксимация кривизны поверхности по сравнению с Гуро;
	\item хорошо передает блики.
\end{itemize}

\subsubsection*{Недостатки:}
\begin{itemize}[label=---]
	\item наиболее трудоемкий алгоритм.
\end{itemize}


\subsection{Вывод}

Для поставленной задачи лучше всего подойдет закраска по Гуро, так как она сможет обеспечить достаточно реалистичное изображение закругленных объектов. Так же этот алгоритм достаточно быстр.


\section{Выбор модели освещения}

Существует две модели освещения, которые используются для построения света на трёхмерных сценах: локальная и глобальная. Исходя из поставленной задачи, необходимо выбрать вариант, который позволит достаточно быстро просчитывать освещения на сцене.

\begin{enumerate}
	\item Локальная модель является самой простой, не рассматривает процессы светового взаимодействия объектов сцены между собой и рассчитывает освещённость только самих объектов.
	\item Глобальная модель рассматривает трёхмерную сцену, как единую систему и описывает освещение с учётом взаимного влияния объектов, что позволяет рассматривать такие явления, как многократное отражение и преломление света, а также рассеянное освещение.
\end{enumerate}

Для поставленной задачи лучше подходит локальная модель освещённости, так как она является более быстродействующей. Также в сцене отсутствуют объекты, обладающие зеркальными или преломляющими свойствами, поэтому использование более качественной глобальной модели не требуется.

Далее будут рассмотрены только локальные модели освещения.


\subsection{Модель Ламберта}

Модель Ламберта~\cite{light} моделирует идеальное диффузное освещение. Свет при попадании на поверхность рассеивается равномерно во все стороны. При расчете такого освещения учитывается только ориентация поверхности (нормаль) и направление на источник света.  

Интенсивность можно рассчитать по формуле~(\ref{eq:lambert}):

\begin{equation}\label{eq:lambert}
	I_{d} = k_{d} \cdot \cos(\overrightarrow{L}, \overrightarrow{N}) \cdot i_{d},
\end{equation}

\noindent где $I_{d}$ --- рассеянная составляющая освещенности в точке, $k_{d}$ --- свойство материала воспринимать рассеянное освещение, $i_{d}$ --- мощность рассеянного освещения,
$\overrightarrow{L}$ --- направление из точки на источник, $\overrightarrow{N}$ --- вектор нормали в точке.


\subsection{Модель Фонга}

Идея модели заключается в предположении, что освещенность каждой точки разлагается на 3 компоненты~\cite{light}: фоновое освещение (ambient), рассеянный свет (diffuse), бликовая составляющая (specular) (рисунок \ref{img:fong}).

\clearpage
\img{40mm}{fong}{Модель освещения Фонга}

Свойства источника определяют мощность излучения для каждой из компонент, а свойства материала --- способность объекта воспринимать свет.

Интенсивность света рассчитывается по формуле~(\ref{eq:fongi}):

\begin{equation}\label{eq:fongi}
	I = k_{a} \cdot I_{a} + k_{d} \cdot (\overrightarrow{N}, \overrightarrow{L}) +  k_{s} \cdot (\overrightarrow{N}, \overrightarrow{V})^p,
\end{equation}

\noindent где $\overrightarrow{N}$ --- вектор нормали к поверхности в точке, $\overrightarrow{L}$ --- направление проецирования (направление на источник света), $\overrightarrow{V}$ --- направление на наблюдателя, $k_{a}$ --- коэффициент фонового освещения, $k_{s}$ --- коэффициент зеркального освещения, $k_{d}$ --- коэффициент диффузного освещения, $p$ --- степень блеска.


\subsection{Вывод}
Для поставленной задачи лучше всего подойдет модель Ламберта, так как программа должна иметь наиболее высокую производительность.


\section{Текстурирование граней объектов с использованием барицентрических координат}

Барицентрические координаты --- это координаты, в которых точка треугольника (грани объекта) описывается как линейная комбинация вершин~\cite{rojers}, то есть произвольная точка треугольника может быть найдена по формуле~(\ref{eq:bar}):
    \begin{equation}\label{eq:bar}
	    m = u \cdot p_{0} + v \cdot p_{1} + w \cdot p_{2},
    \end{equation}

\noindent где $u, v, w$ --- барицентрические координаты, $p_{0}, p_{1}$, $p_{2}$ --- вершины треугольника.

Барицентрические координаты точки неотрицательны и их сумма равна единице.  Первые две координаты равны отношению площадей треугольников, которые
образует точка внутри треугольника и вершины, к общей площади треугольника.
Третья координата вычисляется через две известные.

Барицентрические координаты позволяют интерполировать значение
любого параметра (в том числе и текстуры) в произвольной точке треугольника по формуле~(\ref{eq:bar2}):
    \begin{equation}\label{eq:bar2}
	    t = u \cdot t_{0} + v \cdot t_{1} + w \cdot t_{2},
    \end{equation}

\noindent где $t_{0}, t_{1}$, $t_{2}$ --- значения параметра в вершинах треугольника. 


\section{Физическая модель поведения объектов}

 В наиболее типичных тройных звездах две компоненты обычно образуют тесную бинарную систему (двойную звезду), обращаясь одна вокруг другой на сравнительно небольшом расстоянии, а третья звезда обращается вокруг тесной пары по орбите значительно большего размера. Гравитационное действие тесной пары на удаленного третьего компаньона такое же, каким было бы действие единственной массы. Третья звезда находится так далеко, что ее гравитационное поле не в состоянии повлиять сколько-нибудь значительно на устойчивое относительное движение партнеров внутренней тесной пары.

 Рассмотрим движение звезд в тесной паре. Движение компонент двойных звезд происходит в соответствии с законами Кеплера: оба компонента описывают в пространстве подобные эллиптические орбиты вокруг общего центра масс, который находится в одном из фокусов каждой из орбит. Значения больших полуосей этих эллипсов обратно пропорциональны массам звезд.

 Обращение в системах двойных звезд подчиняется закону всемирного тяготения Ньютона, так как законы Кеплера, как доказал еще сам Ньютон, являются следствием единого закона тяготения, выражающегося формулой~(\ref{eq:f}):

 \begin{equation}\label{eq:f}
	F = G \cdot \frac{m_{1} \cdot m_{2}}{R^2},
\end{equation} 

\noindent где $F$ --- гравитационная сила, действующая между двумя объектами,
$m_{1}$, $m_{2}$ --- массы объектов, $R$ --- расстояние между центрами их масс,
$G$ --- гравитационная постоянная.

Третья более удаленная звезда также движется по эллиптической орбите, в одном из фокусов которой находится центр масс тесной пары.

Теперь рассмотрим вопрос движения планеты, обращающейся вокруг третьей звезды. В данной программе будет моделироваться система, математическая модель которой соответствует ограниченной задаче трех тел: масса одного из тел (планета) пренебрежимо мала по сравнению с массами двух других тел (масса тесной пары и третья звезда). В такой системе можно не принимать во внимание влияние тела малой массы на движение двух других тел.

При описании движения близкой к третьей звезде планеты рассматривается ее движение относительно звезды, а не относительно центра масс тесной пары. Система отсчета, связанная с третьей звездой, не является инерциальной: она подвержена ускорению, направленному к центру масс тесной пары. Так как планета находится близко к своей звезде, гравитационное притяжение тесной пары сообщает ей почти такое же ускорение, как и самой третьей звезде. Поэтому влияние притяжения к тесной паре на движение планеты относительно третьей звезды оказывается незначительным: в главных чертах это движение описывается законами Кеплера. 

В итоге, при расчете движения планеты вблизи третьей звезды оказывается возможным учитывать ее притяжение только этой звездой, т.е. исследовать движение планеты относительно звезды-хозяйки в рамках ограниченной задачи двух тел. Для планеты, масса которой много меньше массы звезды, это будет просто кеплерово движение в ньютоновском поле тяготения. Орбита планеты относительно третьей звезды может быть эллипсом или окружностью.

Итоговая траектория планеты в системе отсчета, связанной с центром масс тесной пары, является результатом сложения двух относительно простых движений: регулярного движения по большому эллипсу вместе с третьей звездой вокруг центра масс тесной пары, и одновременного обращения вокруг третьей звезды по малой окружности (или малому эллипсу)~\cite{moroz}.


\section{Вывод}
В данном разделе был проведен анализ алгоритмов удаления невидимых линий и моделей освещения, которые возможно использовать для построения сцены. В качестве ключевых алгоритмов выбраны алгоритм z-буфера и алгоритм закраски по Гуро. Для представления объектов сцены выбрана поверхностная модель (путем хранения списка граней).

